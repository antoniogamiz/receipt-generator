%%%%%%%%%%%%%%%%%%%%%%%%%%%%%%%%%%%%%%%%%%
%%% NORMALMENTE NO ES NECESARIO HACER 
%%% CAMBIOS EN ESTA PARTE DEL DOCUMENTO
%%%%%%%%%%%%%%%%%%%%%%%%%%%%%%%%%%%%%%%%%%


%:Clase del documento
\documentclass[fontsize=10pt, Myfinal=true, oneside, numbers=noenddot]{report}
%Minion=true, English=true, Myfinal=true

%:Paquete de estilos propuesto
\usepackage[Myfinal=true]{libroETSI}

%:Paquete específico para cargar tikz (y sus librerías) y pgfplots
\usepackage{dtsc-creafig}

%:Paquete para notaciones específicas
\usepackage{notacion}

% Paquete para incluir epígrafes en los capítulos
\usepackage{epigraph}




%:Para modificar fácilmente la fuente del texto.
\makeatletter
\ifdtsc@Minion % Queremos utilizar la fuente Minion y lo hemos declarado al principio
	\ifluatex
		\setmainfont[Renderer=Basic, Ligatures=TeX,	% Fuente del texto 
		Scale=1.01,
		]{Minion Pro}
   		% En este caso conviene modificar ligeramente el tamaño de las fuentes matemáticas
		\DeclareMathSizes{10}{10.5}{7.35}{5.25}
		\DeclareMathSizes{10.95}{11.55}{8.08}{5.77}
		\DeclareMathSizes{12}{12.6}{8.82}{6.3}
%		\setmainfont[Renderer=Basic, Ligatures=TeX,	% Fuente del texto 
%		]{Adobe Garamond Pro}
%		\setmainfont[Renderer=Basic, Ligatures=TeX,	% Fuente del texto 
%		]{Palatino LT Std}
	\fi
\else
	\ifluatex
		% Para utilizar la fuente Times New Roman, o alguna otra que se tenga instalada
		\setmainfont[Renderer=Basic, Ligatures=TeX,	% Fuente del texto 
		Scale=1.0,
		]{Times New Roman}
	\else
		\usepackage{tgtermes} 	%clone of Times
		%\usepackage[default]{droidserif}
		%\usepackage{anttor} 	
	\fi
\fi
\makeatother

% Formato A4
\geometry
{paperheight=297mm,%
paperwidth=210mm,%
top=25mm,%
headsep=8.5mm,%
includefoot, 
textheight=240mm, 
textwidth=150mm, 
bindingoffset=0mm, 
}
%asymmetric----elimina el efecto twoside
%twoside---formato de márgenes a doble cara, estilo libro

\usepackage[a4,center]{crop}%para poner las cruces de esquina de página, poner la opción cross

%:Esquema de numeración por defecto
\setenumerate[1]{label=\normalfont\bfseries{\arabic*.}, leftmargin=*, labelindent=\parindent}
\setenumerate[2]{label=\normalfont\bfseries{\alph*}), leftmargin=*}
\setenumerate[3]{label=\normalfont\bfseries{\roman*.}, leftmargin=*}
\setlist{itemsep=.1em}
\setlength{\parindent}{1.0 em}

\setcounter{tocdepth}{4}						% El nivel hasta el que se muestra el índice 

%\usepackage[framemethod=tikz]{mdframed}

\newcolumntype{M}[1]{>{\centering\arraybackslash}m{#1}}

%Campo de firma
\newcommand{\firmado}[2]{%
\vspace{0.5cm}
  \parbox{\textwidth}{
    \centering #2 \today\\
   % \vspace{4cm}
   
   \begin{center} 
\includegraphics[scale=1, angle=90]{figuras/03-SELLO TESA}
\end{center}    
    \parbox{9cm}{
      \centering
      \rule{6cm}{1pt}\\
       #1 
    }
    \hfill
  }
}

%Para incluir pdf
\usepackage{pdfpages}

%Para poner la marca de agua
\usepackage{eso-pic}
\usepackage{transparent}

\newcommand\BackgroundPic{
\put(0,0){
\parbox[b][\paperheight]{\paperwidth}{%
\vfill
\centering
{\transparent{0.35}\includegraphics[width=13.5cm,height=10cm, angle=90]{figuras/01-LOGO}}%
\vfill
}}}


%\usepackage{lscape} %Sirve para poner en horizontal el texto si vamos a imprimir el PDF
\usepackage{pdflscape} % Pone en horizontal la hoja si no la queremos imprimir
%\AddToShipoutPicture*{\BackgroundPic}

%%%%%%%%%%%%%%%%%%%%%%%%%%%%%%%%%%%%%%%%%%
%%% A PARTIR DE AQUÍ HAY QUE EDITAR
%%%%%%%%%%%%%%%%%%%%%%%%%%%%%%%%%%%%%%%%%%

%:Empieza el documento

\begin{document}


\begin{landscape}
\pagestyle{empty}
\AddToShipoutPicture{\BackgroundPic}	

  \begin{minipage}{\linewidth}
      \begin{minipage}{0.475\linewidth}
\begin{caja}[frametitle=Datos del cliente,userdefinedwidth=8cm,align=center]
	\begin{itemize}
		\item \textbf{Nombre: }!NAME!
		\item \textbf{Dirección: }!ADDRESS!
		\item \textbf{C.P/Ciudad: }!CP!, !CITY!
		\item \textbf{CIF/NIF: }!NIF!
		\item \textbf{Móvil: }!PHONENUMBER!
		\item \textbf{Correo: }!EMAIL!
	\end{itemize}
\end{caja}
      \end{minipage}
      \begin{minipage}{0.475\linewidth}
\begin{caja}[frametitle=Datos del presupuesto,userdefinedwidth=8cm,align=center]
	\begin{itemize}
		\item \textbf{Nº Cliente: }!CLIENTNUMBER!
		\item \textbf{Nº Presupuesto: }!BUDGETNUMBER!
		\item \textbf{Fecha: }\today
	\end{itemize}
\end{caja}
      \end{minipage}
  \end{minipage}

\vspace{0.5cm}

\begin{spacing}{1.2}
\begin{longtable}{M{2cm}M{1.5cm}M{4.5cm}M{1.5cm}M{1.5cm}M{1.5cm}M{2cm}M{2cm}M{2cm}}
\caption{Presupuesto de la instalación detallado} \label{tab01-08}\\
\hline
\hline
\rule[-8pt]{0pt}{0pt}{\bfseries{Referencia}}& {\bfseries{Marca}} & {\bfseries{Descripción}} & {\bfseries{Cantidad}} & {\bfseries{P.prov.}} & {\bfseries{BI(\%)}} & {\bfseries{Precio Unitario}} & {\bfseries{Precio total}} & {\bfseries{Ganancias}}\\
\hline
\hline
& & & & & & & & \\

\hline
\hline
\vspace{0.5cm}
  &        &                                          &    &    &          & Subtotal & !SUBTOTAL! \\
\cline{7-9}
\vspace{0.5cm}
  &        &                                          &    &    &          & GG (13\%) & !GENERALEXPENSES! \\
\cline{7-9} 
\vspace{0.5cm}
  &        &                                          &    &    &          & IVA (21\%) & !IVA! \\
\cline{7-9} 
\vspace{0.5cm}
  &        &                                          &    &    &          & \textbf{Total} & \textbf{!TOTAL1!} & \textbf{!TOTAL2!}\\
\cline{7-9}
%\hline
\end{longtable}
\end{spacing}			 

\end{landscape}

\end{document}