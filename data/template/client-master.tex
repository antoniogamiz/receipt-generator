%%%%%%%%%%%%%%%%%%%%%%%%%%%%%%%%%%%%%%%%%%
%%% NORMALMENTE NO ES NECESARIO HACER 
%%% CAMBIOS EN ESTA PARTE DEL DOCUMENTO
%%%%%%%%%%%%%%%%%%%%%%%%%%%%%%%%%%%%%%%%%%


%:Clase del documento
\documentclass[fontsize=11pt, Myfinal=true, oneside, numbers=noenddot]{report}
%Minion=true, English=true, Myfinal=true

%:Paquete de estilos propuesto
\usepackage[Myfinal=true]{libroETSI}

%:Paquete específico para cargar tikz (y sus librerías) y pgfplots
\usepackage{dtsc-creafig}

%:Paquete para notaciones específicas
\usepackage{notacion}

%:Paquete para incorporar aspectos concretos de la edición
\usepackage{edicionPFC}

% Paquete para incluir epígrafes en los capítulos
\usepackage{epigraph}




%:Para modificar fácilmente la fuente del texto.
\makeatletter
\ifdtsc@Minion % Queremos utilizar la fuente Minion y lo hemos declarado al principio
	\ifluatex
		\setmainfont[Renderer=Basic, Ligatures=TeX,	% Fuente del texto 
		Scale=1.01,
		]{Minion Pro}
   		% En este caso conviene modificar ligeramente el tamaño de las fuentes matemáticas
		\DeclareMathSizes{10}{10.5}{7.35}{5.25}
		\DeclareMathSizes{10.95}{11.55}{8.08}{5.77}
		\DeclareMathSizes{12}{12.6}{8.82}{6.3}
%		\setmainfont[Renderer=Basic, Ligatures=TeX,	% Fuente del texto 
%		]{Adobe Garamond Pro}
%		\setmainfont[Renderer=Basic, Ligatures=TeX,	% Fuente del texto 
%		]{Palatino LT Std}
	\fi
\else
	\ifluatex
		% Para utilizar la fuente Times New Roman, o alguna otra que se tenga instalada
		\setmainfont[Renderer=Basic, Ligatures=TeX,	% Fuente del texto 
		Scale=1.0,
		]{Times New Roman}
	\else
		\usepackage{tgtermes} 	%clone of Times
		%\usepackage[default]{droidserif}
		%\usepackage{anttor} 	
	\fi
\fi
\makeatother

% Formato A4
\geometry
{paperheight=297mm,%
paperwidth=210mm,%
top=25mm,%
headsep=8.5mm,%
includefoot, 
textheight=240mm, 
textwidth=150mm, 
bindingoffset=0mm, 
}
%asymmetric----elimina el efecto twoside
%twoside---formato de márgenes a doble cara, estilo libro

\usepackage[a4,center]{crop}%para poner las cruces de esquina de página, poner la opción cross

%:Esquema de numeración por defecto
\setenumerate[1]{label=\normalfont\bfseries{\arabic*.}, leftmargin=*, labelindent=\parindent}
\setenumerate[2]{label=\normalfont\bfseries{\alph*}), leftmargin=*}
\setenumerate[3]{label=\normalfont\bfseries{\roman*.}, leftmargin=*}
\setlist{itemsep=.1em}
\setlength{\parindent}{1.0 em}

\setcounter{tocdepth}{4}						% El nivel hasta el que se muestra el índice 

%\usepackage[framemethod=tikz]{mdframed}

\newcolumntype{M}[1]{>{\centering\arraybackslash}m{#1}}

%Campo de firma
\newcommand{\firmado}[2]{%
\vspace{0.5cm}
  \parbox{\textwidth}{
    \centering #2 \today\\
   % \vspace{4cm}
   
   \begin{center} 
\includegraphics[scale=1,angle=90]{figuras/03-SELLO TESA}
\end{center}    
    \parbox{9cm}{
      \centering
      \rule{6cm}{1pt}\\
       #1 
    }
    \hfill
  }
}

%Para incluir pdf
\usepackage{pdfpages}

%Para poner la marca de agua
\usepackage{eso-pic}
\usepackage{transparent}

\newcommand\BackgroundPic{
\put(0,-50){
\parbox[b][\paperheight]{\paperwidth}{%
\vfill
\centering
{\transparent{0.35}\includegraphics[width=13.5cm,height=10cm]{figuras/01-LOGO}}%
\vfill
}}}

\usepackage{numprint}
%\AddToShipoutPicture*{\BackgroundPic}

%%%%%%%%%%%%%%%%%%%%%%%%%%%%%%%%%%%%%%%%%%
%%% A PARTIR DE AQUÍ HAY QUE EDITAR
%%%%%%%%%%%%%%%%%%%%%%%%%%%%%%%%%%%%%%%%%%

%:Empieza el documento

\begin{document}
%\selectlanguage{ngerman}
%PORTADA
%ver edicionPFC.sty para modificaciones

%:Para crear la portada y la portada interior (pagina titular)
\nombretrabajo{PROPUESTA Y PRESUPUESTO DE \\INSTALACIÓN FV CONECTADA A RED}
\cliente{Datos del cliente:}
\nombreCliente{!NAME!}
\direccionCliente{!ADDRESS!}
\CP{!CP!, !CITY!}
\CIFNIF{!NIF!}
\emplazamientoInstalacion{!INSTALLATIONADDRESS!}
\Ncliente{!CLIENTNUMBER!}
\Npresupuesto{!BUDGETNUMBER!}
\direccionEmpresa{C/Derechos Humanos, Nº 8, Churriana de la Vega, Granada}
\telefono{(+34) 958176105}
\direccionWeb{www.tesaenergia.com}
\email{tecnico@tesaenergia.com}


\hypersetup
	{
 	linkcolor=black, %Tocar para poner color en enlaces
	pdfauthor={\elnombreCliente},
	pdftitle={\nombretrabajo,\elcliente}, 
	pdfkeywords={Latex, edición, formato de texto}	
	}

\portadaPFC{} %logo de la Universidad y logo del departamento, si lo hubiera. Para cambiar el pie de página con los logos, debe editarse el fichero ediciónPFC.sty

%Fin Portada
\pagestyle{esitscCD}

\chapter{Memoria descriptiva} 
\AddToShipoutPicture{\BackgroundPic}	
\large		 
\begin{spacing}{1.2}
Estimado D./Dña !NAME!,\\
Fue un placer hablar con usted sobre el nuevo proyecto que propone a nuestra empresa. Dicho proyecto trata de realizar una instalación fotovoltaica residencial.

Los objetivos fundamentales de este tipo de instalación son los que se citan a continuación:

\begin{itemize}\itemsep1pt \parskip0pt \parsep0pt
\item Transformar la energía solar procedente del Sol, en energía que se pueda aprovechar en las diferentes tareas del día a día que requieran alimentación eléctrica. 
\item Reducir la factura de la luz un porcentaje considerable con respecto al consumo mensual de la vivienda, antes de realizar dicha instalación.
\item Contribuir en el desarrollo y uso de las energías renovables.
\end{itemize}

Habiendo considerado todos los detalles que se trataron en la primera toma de contacto, le presento esta propuesta formada por una breve memoria descriptiva de los actos, el presupuesto estimado en base a los servicios ofrecidos y un plano de situación de la instalación. Además, se adjuntan las fichas técnicas de los módulos solares e inversores utilizados en la instalación.
				 
\section*{Datos de la instalación}

\begin{itemize}\itemsep1pt \parskip0pt \parsep0pt
\item \textbf{Dirección:} !INSTALLATIONADDRESS! 
\item \textbf{Tipo de instalación:} !INSTALLATIONTYPE!
\item \textbf{Azimut:} !AZIMUT!
\end{itemize}
%A partir de aquí cada presupuesto puede variar
La instalación estará compuesta por un generador fotovoltaico formado por un total de 17 módulos solares. Las características de éstos se resumen a continuación, del mismo modo, se adjunta a este documento su ficha técnica correspondiente.

\begin{itemize}\itemsep1pt \parskip0pt \parsep0pt
\item \textbf{Marca:} KASEEL 
\item \textbf{Modelo:} KSPM-72 PERC. Se trata de un tipo de módulo de alto rendimiento formado por silicio monocristalino, obteniendo una eficiencia más elevada que el resto de módulos del mercado.
\item \textbf{Potencia nominal:} $375$ W$_{pico}$, es decir, esta potencia se consigue en condiciones de laboratorio.
\item \textbf{Garantía:} 91,2\% de su potencia nominal los 12 primeros años, 80,6\% hasta los 30 años.
\end{itemize}
				 
Por lo tanto, el generador fotovoltaico será capaz de generar una potencia pico total P$_{pico total}$, de 6.375 W$_{pico}$.

Debido a que la mayor eficiencia se consigue cuando la radiación solar incide perpendicularmente sobre la superficie de los módulos solares, se propone la fijación de los mismos sobre la cubierta del edificio, mediante estructuras metálicas inclinadas 30º, según se observa en el plano de replanteo que se adjunta. Las características fundamentales de esta estructura son las siguientes:

\begin{itemize}\itemsep1pt \parskip0pt \parsep0pt
\item \textbf{Material:} Aluminio 
\item \textbf{Tratamiento de protección:} Anodización
\item \textbf{Fijación:} Mediante taco químico con protección para evitar filtraciones de agua o placa de hormigón, en caso de tratarse de una superficie horizontal.
\end{itemize}

Se instalará un inversor de acuerdo a la potencia pico del generador fotovoltaico mencionado anteriormente. La función de este componente, es convertir la energía solar en corriente alterna, con las características técnicas de frecuencia (50 Hz) y tensión (230 V) que exige la compañía distribuidora. Además, presenta conexión a internet, por lo que es capaz de mandar información en todo momento sobre la producción del generador fotovoltaico, consumo del hogar, que se obtiene gracias al medidor de energía instalado etc. Las características de este inversor se exponen a continuación, al igual que con los módulos solares, se adjunta la ficha técnica correspondiente.

\begin{itemize}\itemsep1pt \parskip0pt \parsep0pt
\item \textbf{Marca:} KOSTAL 
\item \textbf{Modelo:} PIKO MP PLUS
\item \textbf{Potencia nominal:} 3 kW
\item \textbf{Número de MPPT:} 2
\item \textbf{Garantía:} 5 años
\end{itemize}

Cabe destacar, que la potencia máxima de salida se define por la potencia nominal del inversor, es decir, 3 kW en corriente alterna.

Se utilizarán dos tipos de cables dentro de la instalación:

\begin{itemize}\itemsep1pt \parskip0pt \parsep0pt
\item \textbf{Generador fotovoltaico:} ZZ-F 0,6/1kV XLPE Cu 1x6mm$^2$, se trata de un cable resistente a la interperie, ya que posee un doble recubrimiento. 
\item \textbf{Inversor:} RZ1-K (AS) 0,6/1kV XLPE Cu 1x6mm$^2$, se trata de un cable muy resistente, formado por cobre flexible y recubierto con una capa de aislante.
\end{itemize}

Además de esto, se instalará un cuadro de protecciones, tanto para el generador fotovoltaico, como para el inversor. Entre estas protecciones se encuentran los fusibles, sobretensiones, interruptores magnetotérmicos y diferenciales.

Como se ha comentado anteriormente, se propone la instalación de un medidor de energía. La función de este elemento es leer los datos de consumo de la residencia en todo momento, y enviárselos al inversor. Una vez los datos llegan al inversor, éste los vuelca al servidor web y a la aplicación movil, donde el usuario es capaz de observar los datos de producción solar, consumo, grado de autoconsumo etc. En este caso, se utilizará el medidor de energía SDM 230.

Finalmente, se deberá instalar una toma de tierra, en caso de no exista en el edificio, con la finalidad de proteger la instalación ante cualquier fenómeno anómalo existente.
Además, se procederá a la legalización de la instalación ante industria. Para ello, se deberán instalar todas las medidas de protección que no estén presentes en la instalación existente del edificio, según se especifica en la \textit{Especificación particular NRZ103 de ENDESA "Instalaciones de enlace conectadas a la red de distribución", Edición 2ª, 09-2018} y en el \textit{Reglamento Electrotécnico para Baja Tensión 2002}. En general, estas medidas de protección suelen ser IGA+sobretensiones, módulo de contador tipo CM2 con base portafusibles tipo BUC etc.

\textbf{IMPORTANTE: En esta propuesta de presupuesto no se incluyen los posibles gastos que se pueden originar debido a esta nueva normativa.}
La instalación se realizará siguiendo el siguiente plan cronológico. El plazo de ejecución máximo es de \textbf{10 días desde la recepción de todos los equipos necesarios.}

\begin{itemize}\itemsep1pt \parskip0pt \parsep0pt
\item \textbf{Inicio de la obra:} El inicio de la obra comenzará en el momento que se haga el primer ingreso, como se detalla en el apartado \textit{Presupuesto} de este documento. Una vez recibido el primer pago, se fijará un día para el comienzo de la instalación, dependiendo de la disponibilidad de los materiales en ese momento y del trabajo anterior a éste que se esté realizando. Este periodo no se suele alargar más de \textbf{DOS SEMANAS}. 
\item \textbf{Primer día de trabajo:} En el primer día de trabajo se fijarán los soportes de las estructuras metálicas a la superficie de la cubierta. Estas fijaciones serán tacos químicos o placas de hormigón. Se estudiará por donde irá la canaleta del cableado, y se colocará el inversor.
\item \textbf{Segundo día de trabajo:} Se tirarán todos los cables y se realizarán todas las conexiones necesarias, incluyendo la instalación del cuadro de protecciones.
\item \textbf{Tercer día de trabajo:} Se realizará la configuración del inversor y la puesta en marcha de la instalación. Finalmente, se le configurará la aplicación móvil al cliente para que pueda observar el funcionamiento de su instalación.
\end{itemize}

Si tiene alguna pregunta sobre los precios o necesita añadir o eliminar alguna parte del trabajo, por favor, no dude en ponerse en contacto conmigo, a través de los datos que le facilito:

\begin{itemize}
\item \textbf{Correo electrónico:} tecnico@tesaenergia.com
\item \textbf{Móvil:} $646\,504\,420$ 
\end{itemize}

Agradezco la oportunidad que me dan de enviarles esta propuesta y espero poder trabajar con usted.

\end{spacing}
\endinput

\chapter{Presupuesto}
\AddToShipoutPicture{\BackgroundPic}
%%Marca de agua
%\begin{tikzpicture}[remember picture,overlay]
%		\begin{pgfonlayer}{background}
%				 \node[opacity=0.3,inner sep=0pt,anchor=center,yshift=-50mm,xshift=0mm] at (current page.center){\includegraphics[width=13.5cm,height=10cm]{figuras/01-LOGO}};      
%				 \end{pgfonlayer}
%				 \end{tikzpicture}
%\AddToShipoutPicture{\BackgroundPic}
%Datos
  \begin{minipage}{\linewidth}
      %\centering
      \begin{minipage}{0.475\linewidth}
\begin{caja}[frametitle=Datos del cliente,userdefinedwidth=8cm]
	\begin{itemize}
		\item \textbf{Nombre: }!NAME!
		\item \textbf{Dirección: }!ADDRESS!
		\item \textbf{C.P/Ciudad: }!CP!, !CITY!
		\item \textbf{CIF/NIF: }!NIF!
		\item \textbf{Móvil: }!PHONENUMBER!
		\item \textbf{Correo: }!EMAIL!
	\end{itemize}
\end{caja}
      \end{minipage}
      \hspace{0.05\linewidth}
      \begin{minipage}{0.475\linewidth}
\begin{caja}[frametitle=Datos del presupuesto,userdefinedwidth=7cm]
	\begin{itemize}
		\item \textbf{Nº Cliente: }!CLIENTNUMBER!
		\item \textbf{Nº Presupuesto: }!BUDGETNUMBER!
		\item \textbf{Fecha:}\today
%		\item \textbf{CIF/NIF: }76626884-S
%		\item \textbf{Móvil: }646504420
%		\item \textbf{Correo: }tecnico@tesaenergia.com
	\end{itemize}
\end{caja}
      \end{minipage}
  \end{minipage}

\vspace{0.5cm}


\begin{longtable}{M{2cm}M{1.5cm}M{4.5cm}M{1.5cm}M{1.75cm}M{2cm}}
\caption{Presupuesto de la instalación} \label{tab01-08}\\
\hline
\hline
\rule[-8pt]{0pt}{22pt}{\bfseries{Referencia}}& {\bfseries{Marca}} & {\bfseries{Descripción}} & {\bfseries{Cantidad}} & {\bfseries{Precio Unitario}} & {\bfseries{Precio total}} \\
\hline
\hline
& & & & & \\
\hline
\hline
\vspace{0.5cm}
  &        &                                          &    & Subtotal & !SUBTOTAL! \\
\cline{5-6}
\vspace{0.5cm}
  &        &                                          &    & GG 13\% & !GENERALEXPENSES! \\
\cline{5-6}
\vspace{0.5cm}
  &        &                                          &    & IVA 21\% & !IVA! \\
\cline{5-6}
\vspace{0.5cm}
  &        &                                          &    & \textbf{Total} & \textbf{!TOTAL!} \\
\rule{15.7cm}{2pt}
%\hline
%\hline
\end{longtable}
				 
\section*{Importante:}

\begin{itemize}
\item Este presupuesto tiene una validez de \textbf{10 días laborables.}
\item El plazo de ejecución de la obra será de \textbf{10 días laborables desde la recepción de los equipos}, siempre y cuando se trate de una instalación residencial o menor de 5 kW. En caso contrario dependerá del tamaño de la obra a realizar.
\item Si durante la prestación del servicio esta propuesta se viera ampliada por parte del cliente, este presupuesto quedaría invalidado, y se realizaría uno nuevo en función del trabajo adicional.
\item \textbf{Para confirmar la aceptación de esta propuesta, deberá devolver firmadas, de forma manuscrita o digital, cada una de las hojas que la componen o el documento en general respectivamente. Además, se adjuntará el recibo del primer pago, como se describe en el apartado \textit{Forma de pago}.}
\end{itemize}

\section*{Garantías:}
Como se ha comentado anteriormente, los plazos de garantía de los equipos son los siguientes:
\begin{itemize}
\item Módulos solares: \textbf{91.2\% de su potencia nominal los 12 primeros años}, 80.6\% hasta los 30 años.
\item Inversor: \textbf{2 años, ampliables hasta 5 años totalmente gratis, según la marca de éste}. Se puede ampliar hasta 10 años realizando un pago extra variable según la marca del mismo.
\item Medidor de energía: 2 años.
\item Mantenimiento: En caso de algún fallo producido en la instalación, la empresa se hace cargo desde la puesta en marcha de la misma hasta los 2 años siguientes. En caso de que el/los componente/s afectados se encuentren en garantía, se procederá a la tramitación de la misma, por lo que solo se cobrará la mano de obra necesaria para sustituir dichos componentes. Por el contrarío, si el plazo de garantía se ha cumplido, el cliente recibirá un presupuesto con los gastos de la avería.           
\end{itemize}
\textbf{LA GARANTÍA NO CUBRE DAÑO ALGUNO EN CASO DE MAL USO O ALGUNA MODIFICACIÓN REALIZADA, POSTERIORMENTE A LA PUESTA EN MARCHA, POR ALGUIEN DISTINTO AL TÉCNICO DE LA EMPRESA QUE REALIZÓ LA INSTALACIÓN. }

\section*{Forma de pago:}
				 
\begin{itemize}
\item El importe del presupuesto se pagará de la siguiente manera:
\begin{itemize}
\item 50\% en el momento de la aceptación del presupuesto.
\item 50\% restante, durante los 7 días siguientes de finalizar la instalación, o en el mismo día que se entrega toda la documentación de la legalización de la instalación. 
\end{itemize}
\item El pago se realizará por transferencia bancaria: \textbf{ES96 0182 7086 7302 0157 0813}
\end{itemize}

\firmado{Carlos Salcedo Ruíz}{Churriana de la Vega (Granada),}

\vspace{1cm}
\small
\textit{TESLA ENERGY SALCEDO S.L} es el Responsable del tratamiento de los datos personales proporcionados bajo su consentimiento y le informa que estos datos serán tratados de conformidad con lo dispuesto en el Reglamento (UE) 2016/679, de 27 de abril de 2016 (RGPD), siendo su finalidad la de mantener una mutua relación comercial con usted y conservarlos mientras exista un interés mutuo para mantener el fin del tratamiento y, cuando ya no sea necesario para tal fin, se suprimirán con medidas de seguridad adecuadas para garantizar la seudonimización de los datos o la destrucción total de los mismos. No se comunicarán los datos a terceros, salvo obligación legal. Asimismo, se informa que puede ejercer los derechos de acceso, rectificación, portabilidad y supresión de sus datos y los de limitación y oposición a su tratamiento, dirigiéndose a \textit{TESLA ENERGY SALCEDO S.L}, bien por vía postal, en C/ Derechos Humanos, Nº8, 18194, Churriana de la Vega, Granada, o bien por email: \underline{contable@tesaenergia.com}

!INPUTFILES!


\end{document}