
\chapter{Memoria descriptiva} 
\AddToShipoutPicture{\BackgroundPic}	
\large		 
\begin{spacing}{1.2}
Estimado D./Dña !NAME!,\\
Fue un placer hablar con usted sobre el nuevo proyecto que propone a nuestra empresa. Dicho proyecto trata de realizar una instalación fotovoltaica residencial.

Los objetivos fundamentales de este tipo de instalación son los que se citan a continuación:

\begin{itemize}\itemsep1pt \parskip0pt \parsep0pt
\item Transformar la energía solar procedente del Sol, en energía que se pueda aprovechar en las diferentes tareas del día a día que requieran alimentación eléctrica. 
\item Reducir la factura de la luz un porcentaje considerable con respecto al consumo mensual de la vivienda, antes de realizar dicha instalación.
\item Contribuir en el desarrollo y uso de las energías renovables.
\end{itemize}

Habiendo considerado todos los detalles que se trataron en la primera toma de contacto, le presento esta propuesta formada por una breve memoria descriptiva de los actos, el presupuesto estimado en base a los servicios ofrecidos y un plano de situación de la instalación. Además, se adjuntan las fichas técnicas de los módulos solares e inversores utilizados en la instalación.
				 
\section*{Datos de la instalación}

\begin{itemize}\itemsep1pt \parskip0pt \parsep0pt
\item \textbf{Dirección:} !INSTALLATIONADDRESS! 
\item \textbf{Tipo de instalación:} !INSTALLATIONTYPE!
\item \textbf{Azimut:} !AZIMUT!
\end{itemize}
%A partir de aquí cada presupuesto puede variar
La instalación estará compuesta por un generador fotovoltaico formado por un total de 17 módulos solares. Las características de éstos se resumen a continuación, del mismo modo, se adjunta a este documento su ficha técnica correspondiente.

\begin{itemize}\itemsep1pt \parskip0pt \parsep0pt
\item \textbf{Marca:} KASEEL 
\item \textbf{Modelo:} KSPM-72 PERC. Se trata de un tipo de módulo de alto rendimiento formado por silicio monocristalino, obteniendo una eficiencia más elevada que el resto de módulos del mercado.
\item \textbf{Potencia nominal:} $375$ W$_{pico}$, es decir, esta potencia se consigue en condiciones de laboratorio.
\item \textbf{Garantía:} 91,2\% de su potencia nominal los 12 primeros años, 80,6\% hasta los 30 años.
\end{itemize}
				 
Por lo tanto, el generador fotovoltaico será capaz de generar una potencia pico total P$_{pico total}$, de 6.375 W$_{pico}$.

Debido a que la mayor eficiencia se consigue cuando la radiación solar incide perpendicularmente sobre la superficie de los módulos solares, se propone la fijación de los mismos sobre la cubierta del edificio, mediante estructuras metálicas inclinadas 30º, según se observa en el plano de replanteo que se adjunta. Las características fundamentales de esta estructura son las siguientes:

\begin{itemize}\itemsep1pt \parskip0pt \parsep0pt
\item \textbf{Material:} Aluminio 
\item \textbf{Tratamiento de protección:} Anodización
\item \textbf{Fijación:} Mediante taco químico con protección para evitar filtraciones de agua o placa de hormigón, en caso de tratarse de una superficie horizontal.
\end{itemize}

Se instalará un inversor de acuerdo a la potencia pico del generador fotovoltaico mencionado anteriormente. La función de este componente, es convertir la energía solar en corriente alterna, con las características técnicas de frecuencia (50 Hz) y tensión (230 V) que exige la compañía distribuidora. Además, presenta conexión a internet, por lo que es capaz de mandar información en todo momento sobre la producción del generador fotovoltaico, consumo del hogar, que se obtiene gracias al medidor de energía instalado etc. Las características de este inversor se exponen a continuación, al igual que con los módulos solares, se adjunta la ficha técnica correspondiente.

\begin{itemize}\itemsep1pt \parskip0pt \parsep0pt
\item \textbf{Marca:} KOSTAL 
\item \textbf{Modelo:} PIKO MP PLUS
\item \textbf{Potencia nominal:} 3 kW
\item \textbf{Número de MPPT:} 2
\item \textbf{Garantía:} 5 años
\end{itemize}

Cabe destacar, que la potencia máxima de salida se define por la potencia nominal del inversor, es decir, 3 kW en corriente alterna.

Se utilizarán dos tipos de cables dentro de la instalación:

\begin{itemize}\itemsep1pt \parskip0pt \parsep0pt
\item \textbf{Generador fotovoltaico:} ZZ-F 0,6/1kV XLPE Cu 1x6mm$^2$, se trata de un cable resistente a la interperie, ya que posee un doble recubrimiento. 
\item \textbf{Inversor:} RZ1-K (AS) 0,6/1kV XLPE Cu 1x6mm$^2$, se trata de un cable muy resistente, formado por cobre flexible y recubierto con una capa de aislante.
\end{itemize}

Además de esto, se instalará un cuadro de protecciones, tanto para el generador fotovoltaico, como para el inversor. Entre estas protecciones se encuentran los fusibles, sobretensiones, interruptores magnetotérmicos y diferenciales.

Como se ha comentado anteriormente, se propone la instalación de un medidor de energía. La función de este elemento es leer los datos de consumo de la residencia en todo momento, y enviárselos al inversor. Una vez los datos llegan al inversor, éste los vuelca al servidor web y a la aplicación movil, donde el usuario es capaz de observar los datos de producción solar, consumo, grado de autoconsumo etc. En este caso, se utilizará el medidor de energía SDM 230.

Finalmente, se deberá instalar una toma de tierra, en caso de no exista en el edificio, con la finalidad de proteger la instalación ante cualquier fenómeno anómalo existente.
Además, se procederá a la legalización de la instalación ante industria. Para ello, se deberán instalar todas las medidas de protección que no estén presentes en la instalación existente del edificio, según se especifica en la \textit{Especificación particular NRZ103 de ENDESA "Instalaciones de enlace conectadas a la red de distribución", Edición 2ª, 09-2018} y en el \textit{Reglamento Electrotécnico para Baja Tensión 2002}. En general, estas medidas de protección suelen ser IGA+sobretensiones, módulo de contador tipo CM2 con base portafusibles tipo BUC etc.

\textbf{IMPORTANTE: En esta propuesta de presupuesto no se incluyen los posibles gastos que se pueden originar debido a esta nueva normativa.}
La instalación se realizará siguiendo el siguiente plan cronológico. El plazo de ejecución máximo es de \textbf{10 días desde la recepción de todos los equipos necesarios.}

\begin{itemize}\itemsep1pt \parskip0pt \parsep0pt
\item \textbf{Inicio de la obra:} El inicio de la obra comenzará en el momento que se haga el primer ingreso, como se detalla en el apartado \textit{Presupuesto} de este documento. Una vez recibido el primer pago, se fijará un día para el comienzo de la instalación, dependiendo de la disponibilidad de los materiales en ese momento y del trabajo anterior a éste que se esté realizando. Este periodo no se suele alargar más de \textbf{DOS SEMANAS}. 
\item \textbf{Primer día de trabajo:} En el primer día de trabajo se fijarán los soportes de las estructuras metálicas a la superficie de la cubierta. Estas fijaciones serán tacos químicos o placas de hormigón. Se estudiará por donde irá la canaleta del cableado, y se colocará el inversor.
\item \textbf{Segundo día de trabajo:} Se tirarán todos los cables y se realizarán todas las conexiones necesarias, incluyendo la instalación del cuadro de protecciones.
\item \textbf{Tercer día de trabajo:} Se realizará la configuración del inversor y la puesta en marcha de la instalación. Finalmente, se le configurará la aplicación móvil al cliente para que pueda observar el funcionamiento de su instalación.
\end{itemize}

Si tiene alguna pregunta sobre los precios o necesita añadir o eliminar alguna parte del trabajo, por favor, no dude en ponerse en contacto conmigo, a través de los datos que le facilito:

\begin{itemize}
\item \textbf{Correo electrónico:} tecnico@tesaenergia.com
\item \textbf{Móvil:} $646\,504\,420$ 
\end{itemize}

Agradezco la oportunidad que me dan de enviarles esta propuesta y espero poder trabajar con usted.

\end{spacing}
\endinput
